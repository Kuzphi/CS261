\normalfont\documentclass[letterpaper,11pt]{article}
\usepackage{amsmath, amsfonts,amssymb,latexsym}
\usepackage{fullpage}
\usepackage{parskip}
\usepackage{flexisym}
\usepackage{algorithm}
\usepackage{indentfirst}
\usepackage{graphicx}
\usepackage{algorithmicx}
\usepackage{algpseudocode}
\usepackage{pythonhighlight}
\usepackage{amsmath}
\usepackage{hyperref}
\begin{document}
\setlength{\parindent}{2ex}
\newcommand{\header}{
	\noindent \fbox{
	\begin{minipage}{6.4in}
  	\medskip
  	\textbf{CS 261 - Data Structure} \hfill \textbf{Spring 2017} \\[1mm]
  	\begin{center}
    	{\Large HomeWork 6} \\[3mm]
  	\end{center}
	\today \hfill \itshape{Liangjian Chen}
	\medskip
	\end{minipage}}
}
\newcommand{\RN}[1]{%
  \textup{\uppercase\expandafter{\romannumeral#1}}%
}
\bigskip
\header

\begin{enumerate}
\item [Problem 1]\textbf{Solution:}\par
For every node, we record how many leaves in its subtree.\par
Then we found the corresponding node of $q$, the number of leaves is same as the numebr of match postion.\par

\item [Problem 2]\textbf{Solution:}\par
Assume tree size is $n$ and it is $0$-index.\par
If $n$ is an odd number, unrank($\lfloor\frac{n}{2}\rfloor$) will return the answer.\par
Otherwise, the median of unrank($\lfloor\frac{n}{2}\rfloor$) and unrank($\lfloor\frac{n}{2}\rfloor - 1$) is the answer.

\item [Problem 3]\textbf{Solution:}\par
\begin{enumerate}
	\item[$s > t$:]\par
	when $s > t$, $s$ operations at most change $s$ different memory cells. Thus the lower bound is $\Omega(s\log s)$.
	\item[$s < t$:]\par
	Assume $x = t \mod s$.\par
	Decompsing $t$ opeations into several different group each with size $s$ and a extra one group with size $x$. Then for group with size s, it is $\Omega(s\log s)$. For the group with size $x$, from $s < t$, we konw it is $\Omega(x\log x)$. Thus the final answer is $\lfloor\frac{t}{s}\rfloor * s\log s + x\log x = \Omega(s\log t)$
	Therefore it is $\min(s\log s, s \log t)$
\end{enumerate}
\item [Problem 4]\textbf{Solution:}\par
\begin{flalign*}
	&\sum_{x=1}^{i}\sum_{y = x}^{i}2^y =\sum_{x=1}^i{x2^x} = (i-1)2^{i+1} + 2&
\end{flalign*}
\end{enumerate}
\end{document}
